%
% Niniejszy plik stanowi przykład formatowania pracy magisterskiej na
% Wydziale MIM UW.  Szkielet użytych poleceń można wykorzystywać do
% woli, np. formatujac wlasna prace.
%
% Zawartosc merytoryczna stanowi oryginalnosiagniecie
% naukowosciowe Marcina Wolinskiego.  Wszelkie prawa zastrzeżone.
%
% Copyright (c) 2001 by Marcin Woliński <M.Wolinski@gust.org.pl>
% Poprawki spowodowane zmianami przepisów - Marcin Szczuka, 1.10.2004
% Poprawki spowodowane zmianami przepisow i ujednolicenie 
% - Seweryn Karłowicz, 05.05.2006
% Dodanie wielu autorów i tłumaczenia na angielski - Kuba Pochrybniak, 29.11.2016

% dodaj opcję [licencjacka] dla pracy licencjackiej
% dodaj opcję [en] dla wersji angielskiej (mogą być obie: [licencjacka,en])
\documentclass[]{pracamgr}


% Dane magistranta:
\autor{Jakub Bujak}{370737}

\title{Logika separacji dla języka programowania Jafun}

%kierunek: 
% - matematyka, informacyka, ...
% - Mathematics, Computer Science, ...
\kierunek{informatyka}

% informatyka - nie okreslamy zakresu (opcja zakomentowana)
% matematyka - zakres moze pozostac nieokreslony,
% a jesli ma byc okreslony dla pracy mgr,
% to przyjmuje jedna z wartosci:
% {metod matematycznych w finansach}
% {metod matematycznych w ubezpieczeniach}
% {matematyki stosowanej}
% {nauczania matematyki}
% Dla pracy licencjackiej mamy natomiast
% mozliwosc wpisania takiej wartosci zakresu:
% {Jednoczesnych Studiow Ekonomiczno--Matematycznych}

% \zakres{Tu wpisac, jesli trzeba, jedna z opcji podanych wyzej}

% Praca wykonana pod kierunkiem:
% (podać tytuł/stopień imię i nazwisko opiekuna
% Instytut
% ew. Wydział ew. Uczelnia (jeżeli nie MIM UW))
\opiekun{dr hab. Aleksego Schuberta, prof. UW}

% miesiąc i~rok:
\date{Wrzesień 2020}

%Podać dziedzinę wg klasyfikacji Socrates-Erasmus:
\dziedzina{ 
%11.0 Matematyka, Informatyka:\\ 
%11.1 Matematyka\\ 
%11.2 Statystyka\\ 
11.3 Informatyka\\ 
%11.4 Sztuczna inteligencja\\ 
%11.5 Nauki aktuarialne\\
%11.9 Inne nauki matematyczne i informatyczne
}

%Klasyfikacja tematyczna wedlug AMS (matematyka) lub ACM (informatyka)
\klasyfikacja{}

% Słowa kluczowe:
\keywords{Logika separacji, Jafun, weryfikacja}

% Tu jest dobre miejsce na Twoje własne makra i~środowiska:
\newtheorem{defi}{Definicja}[section]

\usepackage{amsmath}
\let\lll\undefined
\usepackage{amssymb}

\newcommand \wand {\mathrel{-\mkern-6mu*}}
\newcommand \outerP {\mathbf{P}}
\newcommand \hoare [5] {\{#1\}#2\{#3,#4.#5\}}
\renewcommand \| {\hspace{0.75em} | \hspace{0.75em} }
\renewcommand \[ {[\![}
\renewcommand \] {]\!]}

% koniec definicji

\begin{document}

\maketitle

%tu idzie streszczenie na strone poczatkowa
\begin{abstract}
  W~pracy zdefiniowano logikę separacji dla języka Jafun, przedstawiono jej formalizację
  w~systemie Coq i~~udowoniono jej poprawność względem semantyki języka. Logika separacji pozwala
  na~podział sterty na rozłączne fragmenty. Upraszcza to~wnioskowanie o~programach, pozwalając
  na~dowodzenie własności podwyrażeń na~prostszych fragmentach sterty.
\end{abstract}

\tableofcontents
%\listoffigures
%\listoftablesC1 1 A

\chapter*{Wprowadzenie}
\addcontentsline{toc}{chapter}{Wprowadzenie}

\chapter{Podstawowe pojęcia i definicje}

\chapter{Jafun}

\section{Składnia i semantyka}

\section{Ewaluacja}
Ewaluacją konfiguracji $(h, st)$ będziemy nazywać dowolny ciąg par $(h_1, st_1), \ldots, (h_n, st_n)$, taki że
$h_1 = h$, $st_1 = st$ oraz $(h_i, st_i) \rightarrow (h_{i+1}, st_{i+1})$ dla $1 \leq i < n$.

Ewaluacją wyrażenia $e$ na stercie $h$ będziemy nazywać taką ewaluację konfiguracji $(h, \[ e \]_\phi)$,
że $st_n = \[ l \]_A$ dla pewnych $l, A$. Jeśli taka ewaluacja istnieje, będziemy to oznaczać jako
$(h, e) \leadsto (h_n, A, l)$

\chapter{Składnia i semantyka}
Prezentowana logika separacji dla języka Jafun jest logiką z~kwantyfikatorami egzystencjalnymi pierwszego rzędu, trójkami Hoare'a, operatorem
pozwalającym na opisywanie zawartości sterty i~operatorami separacji $*$ i $\wand$.

Iris, na~którym wzorowana jest niniejsza logika, jest afiniczną logiką separacyjną, to~znaczy
własność spełniania termu przez stertę jest domknięta ze względu na rozszerzanie sterty.
W celu zachowania zarówno afiniczność, jak i~poprawności względem semantyki
języka, logika nie zawiera kwantyfikatora ogólnego, a kwantyfikator egzystencjalny jest ograniczony
do~termów najwyższego poziomu (Rysunek \ref{fig:syntax}).

\begin{figure}[h]
\begin{align*}
 \outerP ::= & \ \exists x : C.\outerP \| \outerP \wedge \outerP \| \outerP \vee \outerP \| P \\
 P ::= & \ \mathtt{True} \| \mathtt{False} \| P \wedge P \| P \vee P \| P \Rightarrow P \| v = v \| \\
     & \  v.x = v \| \hoare{P}{e}{A}{x}{P} \|  P * P \| P \wand P \\
 v ::= & \  x \| \mathtt{null} \| \mathtt{this} \\
 A ::= & \ C \| \phi \\
 x ::= & \ \langle\mathit{identifier}\rangle \ (\mathit{variable / field\ name}) \\
 C ::= & \ \langle\mathit{identifier}\rangle \ (\mathit{class\ name}) \\
 e ::= & \ \langle \mathit{Jafun\ expression} \rangle
\end{align*}
\caption{Składnia logiki}
\label{fig:syntax}
\end{figure}

Środowisko to~funkcja częściowa przypisująca identyfikatorom lokacje na stercie lub \texttt{null}.
Semantyka logiki (Rysunek \ref{fig:sematics}) jest standardowa dla kwantyfikatora i operatorów logicznych.
Dla uproszczenia zapisu notacja $\[ \cdot \]$ została użyta do~opisu semantyki obu poziomów termów
($\outerP$ i $P$). To, do którego poziomu się odnosi wynika z kontekstu.

Sterta spełnia trójkę Hoare'a $\hoare{P}{e}{A}{x}{Q}$, jeśli dla~każdej sterty spełniającej
$P$, wyrażenie $e$ zostanie obliczone bez błędu, zwróci wyjątek typu $A$,
a~wynikowa sterta będzie
spełniała $Q$, w którym za~$x$~podstawiony zostanie wynik obliczenia.

Sterta spełnia term $P * Q$, jeśli można ją podzielić na dwa rozłączne fragmenty, z których
jeden spełnia $P$, a drugi $Q$.
Operator $\wand$ to pewnego rodzaju odwrotność operatora $*$ -- sterta spełnia $P \wand Q$, jeśli
po połączeniu jej z dowolną rozłączną stertą spełniającą $P$, otrzymana stera spełnia $Q$.

Sterta \textit{h} spełnia term $\outerP$ w~środowisku \textit{env} (zapis $h, env \models \outerP $),
jeśli $\[ \outerP \]_{h, env} = \top$.

\begin{figure}
\begin{align*}
 \[ \texttt{True} \] \triangleq & \ \top \\
 \[ \texttt{False} \] \triangleq & \ \bot \\
 \[ \exists x : C . P \]_{h, env} \triangleq & \ \exists l : Loc\ .\ class(h, l) = C \wedge
    \[ P \]_{h, env[x \mapsto l]}  \\
 \[ P \wedge Q \]_{h, env} \triangleq & \ \[ P \]_{h, env} \wedge \[ Q \]_{h, env} \\
 \[ P \vee Q \]_{h, env} \triangleq & \ \[ P \]_{h, env} \vee \[ Q \]_{h, env} \\
 \[ P \Rightarrow Q \]_{h, env} \triangleq & \ \[ P \]_{h, env} \Rightarrow \[ Q \]_{h, env} \\
 \[ x = y \]_{h, env} \triangleq & \ env(x) = env(y) \\
 \[ x.f = y \]_{h, env} \triangleq & \ h(env(x))(f) = env(y) \\
 \[ \hoare{P}{e}{A}{x}{Q} \]_{h, env} \triangleq & \forall h : Heap \ . \ \[ P \]_{h, env} \Rightarrow \\
      & \ \  \exists h' : Heap, l : Loc \ . \ (h, e[/env]) \leadsto (h', A, l) \wedge
      \[ Q \]_{h', env[x \mapsto l]}  \\
 \[ P * Q \]_{h, env} \triangleq & \ \exists h_1,h_2 : Heap\ .\ h_1 \oplus h_2 = h \wedge
    \[ P \]_{h_1, env} \wedge \[ Q \]_{h2_, env}  \\
 \[ P \wand Q \]_{h, env} \triangleq & \forall h' : Heap \ . \ \[ P \]_{h', env} \Rightarrow \[ Q \]_{h \oplus h', env}
\end{align*}

Uwaga: $e[/env]$ oznacza wyrażenie powstałe przez~podstawienie $env[x]$ w~miejsce $x$ dla~każdej zmiennej
wolnej $x$ w~$e$.

\caption{Semantyka logiki}
\label{fig:sematics}
\end{figure}

\chapter{Reguły wnioskowania}

\chapter{Własności ewaluacji}

\chapter{Poprawność}

\chapter{Formalizacja w systemie Coq}

\chapter{Podsumowanie}

\appendix
\begin{thebibliography}{99}
\addcontentsline{toc}{chapter}{Bibliografia}

\bibitem[Bea65]{beaman} Juliusz Beaman, \textit{Morbidity of the Jolly
    function}, Mathematica Absurdica, 117 (1965) 338--9.

\end{thebibliography}

\end{document}


%%% Local Variables:
%%% mode: latex
%%% TeX-master: t
%%% coding: latin-2
%%% End:
